%* ===============================================================
% <+#TITLE#+>
% <+#SHORTTITLE#+>
% <+#WHEN#+>
% <+#WHERE#+>
%  == Copyright Sergio J. Rey <+$YEAR$;R+++++++++++++++++++> ==
%  =============================================================== */


\documentclass{beamer}
  %#comment out the Boadilla theme and uses only the header bar
  \usetheme[]{Boadilla} 
  %\usetheme[secheader]{Boadilla}

    %#make sure to change this part, since it is predefined
    %\defbeamertemplate*{footline}{infolines theme}
    \setbeamertemplate{footline}
      {
    \leavevmode%
    \hbox{%
    \begin{beamercolorbox}[wd=.333333\paperwidth,ht=2.25ex,dp=1ex,center]{author in head/foot}%
      \usebeamerfont{author in head/foot}\insertshortauthor~~(\insertshortinstitute)
    \end{beamercolorbox}%
    \begin{beamercolorbox}[wd=.333333\paperwidth,ht=2.25ex,dp=1ex,center]{title in head/foot}%
      \usebeamerfont{title in head/foot}\insertshorttitle
    \end{beamercolorbox}%
    \begin{beamercolorbox}[wd=.333333\paperwidth,ht=2.25ex,dp=1ex,right]{date in head/foot}%
      \usebeamerfont{date in head/foot}\insertshortdate{}\hspace*{2em}

  %%turning the next line into a comment, erases the frame numbers
      %\insertframenumber{} / \inserttotalframenumber\hspace*{2ex} 
    \end{beamercolorbox}}%
    \vskip0pt%
  }

\usepackage{multirow}
%%%%%%%%%%%%%%%%%%%%%%%%%%%%%%%%%%%%%%%%
% new content goes below here
%%%%%%%%%%%%%%%%%%%%%%%%%%%%%%%%%%%%%%%%

\usepackage{fancyvrb}
\usepackage{listings}
\usepackage{tikz}
  \graphicspath{{figures/}}

  \usepackage{tikz,overpic}
  \usetikzlibrary{fit,shapes.misc}
\author{Sergio Rey}
\institute[ASU]{GeoDa Center for Geospatial Analysis and Computation\\School of Geographical Sciences and Urban Planning\\Arizona State University}
\title[<+#SHORTTITLE#+>]{<+#TITLE#+>}
\subtitle{}
\date[<+#WHEN#+>]{<+#WHERE#+>}

% Delete this, if you do not want the table of contents to pop up at
% the beginning of each subsection:
\AtBeginSubsection[]
{
  \begin{frame}<beamer>
    \frametitle{Outline}
    \tableofcontents[currentsection,currentsubsection]
  \end{frame}
}


% If you wish to uncover everything in a step-wise fashion, uncomment
% the following command: 
%\beamerdefaultoverlayspecification{<+->}
\begin{document}
\begin{frame}
  \titlepage
\end{frame}

\begin{frame}\frametitle{Outline}
\tableofcontents
\end{frame}

\section{Background} 
\begin{frame}\frametitle{Motivation}
  \pause
  \begin{block}{Tobler's Law}
    "Everything is related to everything else, but near things are more related than distant things."
  \end{block}
  \pause
  \begin{block}{Spatial Autocorrelation}
    Nonrandom attribute value similarity or dissimilarity in geographic space.
  \end{block}
\pause
  \begin{block}{Goodchild 2002}
    ``Hell is a place with no spatial dependence.''
  \end{block}

\end{frame}

\begin{frame}
  \begin{center}
       \begin{figure}
       \scalebox{1.00}{
       \includegraphics[width=1.0\linewidth]{sarcor} }
        \end{figure}      
  \end{center}

\end{frame}

\begin{frame}
  \begin{center}
       \begin{figure}
       \scalebox{1.00}{
       \includegraphics[width=1.0\linewidth]{irregular} }
        \end{figure}      
  \end{center}

\end{frame}


\begin{frame}\frametitle{Acknowledgments}
\begin{itemize}
  \item \textbf{NSF}  CyberGIS Software Integration for Sustained Geospatial Innovation
  \item \textbf{NIJ}  Flexible Geospatial Visual Analytics and Simulation Technologies to Enhance Criminal Justice Decision Support Systems
  \item \textbf{NIH} Geospatial Factors and Impacts: Measurement and Use 
  \item \textbf{NSF} Spatial Analytical Framework for Examining Sex Offender
    Residency Issues Over Space and Time
  \item \textbf{NSF} An Exploratory Space-Time Data Analysis Toolkit for Spatial Social Science Research 
  \item \textbf{NSF}  Hedonic Models of Location Decisions with Applications to Geospatial Microdata 
\end{itemize}
\end{frame}



\begin{frame}\frametitle{Team}
  \begin{center}
 \begin{tabular}{ll}
Serge Rey& Luc Anselin\\
Charles Schmidt & Dave Folch\\
Myunghwa Hwang& Dani Arribas\\
Phil Stephens&Julia Koschinsky\\
Pedro Amaral&Nick Malizia\\
Xing Kang&Xun Li\\
Xinyue Ye&Andrew Winslow\\
Mark McCann& Ran Wei\\
Nancy Lozano & Jing Yao\\
\end{tabular}
  \end{center}
and contributions from many others!
 
\end{frame}
\subsection{Origins}
\begin{frame}\frametitle{PySAL Objectives}
  \begin{block}{Leverage Existing Tools Development}
    \begin{itemize}
      \item GeoDa/PySpace
      \item STARS
    \end{itemize}
  \end{block}
  \begin{block}{Develop Core Library}
    \begin{itemize}
      \item spatial data \emph{analytical} functions
      \item enhanced specialization, modularity
      \item fill void in geospatial Python libraries
    \end{itemize}
  \end{block}
  \begin{block}{Flexible Delivery Mechanisms}
    \begin{itemize}
      \item interactive shell
      \item GUI
      \item Toolkits
      \item webservices
    \end{itemize}
  \end{block}
\end{frame}

\subsection{Progress}

\begin{frame}\frametitle{Since 2009: http://pysal.org}
    \begin{center}
         \begin{figure}
           \scalebox{1.00}{
           \includegraphics[width=1.0\linewidth]{progress09} }
          \end{figure}      
    \end{center}
\end{frame}

\section{Components}

\subsection{Overview}

%last time - weights overview and demo
%last time - autocorrelation view

\begin{frame}\frametitle{Components}
\begin{center}
     \begin{figure}
     \scalebox{0.75}{
     \includegraphics[width=1.0\linewidth]{pysalGraphic} }
      \end{figure}      
\end{center}
  
\end{frame}



\subsection{Spatial Weights}

\begin{frame}\frametitle{Spatial Weights}
  \begin{columns}
  \begin{column}[l]{5cm}
    \begin{block}{Types}
    \begin{itemize}
      \item Contiguity
      \item Distance Based
      \item Kernel Weights
      \item Regimes
      \item Hybrid
    \end{itemize}
  \end{block}
  \end{column}
  \begin{column}[r]{5cm}
    \begin{block}{Operations}
    \begin{itemize}
      \item Standardizations
       \item Higher orders
       \item Summary properties
       \item Set operations
       \item Conversions
    \end{itemize}
  \end{block}
  \end{column}
  \end{columns}
\end{frame}

\begin{frame}[fragile]
  \frametitle{Supported Weights Formats}
  \begin{footnotesize}
  \begin{Verbatim}[numbers=left]
In [7]: pysal.open.check()
PySAL File I/O understands the following file extensions:
Ext: '.shp', Modes: ['r', 'wb', 'w', 'rb']
Ext: '.mtx', Modes: ['r', 'w']
Ext: '.swm', Modes: ['r', 'w']
Ext: '.mat', Modes: ['r', 'w']
Ext: '.shx', Modes: ['r', 'wb', 'w', 'rb']
Ext: '.stata_text', Modes: ['r', 'w']
Ext: '.geoda_txt', Modes: ['r']
Ext: '.dbf', Modes: ['r', 'w']
Ext: '.dat', Modes: ['r', 'w']
Ext: '.gwt', Modes: ['r', 'w']
Ext: '.gal', Modes: ['r', 'w']
Ext: '.arcgis_text', Modes: ['r', 'w']
Ext: '.wk1', Modes: ['r', 'w']
Ext: '.arcgis_dbf', Modes: ['r', 'w']
Ext: '.geobugs_text', Modes: ['r', 'w']
Ext: '.csv', Modes: ['r']
Ext: '.wkt', Modes: ['r']
\end{Verbatim} 
\end{footnotesize}
\end{frame}

\begin{frame}\frametitle{Shapefiles}
\begin{center}
     \begin{figure}
     \scalebox{0.95}{
     \includegraphics[width=1.0\linewidth]{south_shapefile} }
      \end{figure}      
\end{center}
  
\end{frame}



\begin{frame}[fragile]
  \frametitle{W from Shapefiles}
  \begin{footnotesize}
  \begin{Verbatim}[numbers=left]
>>> import pysal
>>> w = pysal.rook_from_shapefile('south.shp')
>>> w.n
1412
>>> w.s0
7700.0
>>> w.n**2
1993744
>>> w.pct_nonzero
0.0038620805880795125
>>> wq = pysal.queen_from_shapefile('south.shp')
>>> wq.s0
8096.0
>>> w.histogram
[(1, 16), (2, 32), (3, 65), (4, 187), (5, 378), (6, 435), (7, 230),
(8, 56), (9, 11), (10, 2)]
>>> wq.histogram
[(1, 16), (2, 29), (3, 54), (4, 141), (5, 306), (6, 437), (7, 319),
(8, 86), (9, 18), (10, 5), (11, 1)]
\end{Verbatim} 
\end{footnotesize}
\end{frame}




\subsection{Spatial Dynamics}
\begin{frame}\frametitle{Space-Time EDA}
  \begin{block}{Spatial pattern of dynamics}
    \begin{itemize}
      \item Temporal co-movements
      \item Spatial clustering of dyads (and beyond)
      \item Optimal currency areas
       \item Convergence clubs, spatial poverty traps
    \end{itemize}
  \end{block}
  \begin{block}{Temporal stability of spatial patterns}
    \begin{itemize}
      \item Choropleth map and raster comparisons
      \item Emergent properties (ABM)
      \item Hot and cold spot detection
    \end{itemize}
  \end{block}
\end{frame}

\begin{frame}[fragile]\frametitle{Spatial Markov }
  \begin{footnotesize}
  \begin{Verbatim}[numbers=left]
>>> import pysal
>>> f = pysal.open(pysal.examples.get_path("usjoin.csv"))
>>> pci = np.array([f.by_col[str(y)] for y in range(1929,2010)])
>>> pci = pci.transpose()
>>> rpci = pci/(pci.mean(axis=0))
>>> w = pysal.open(pysal.examples.get_path("states48.gal")).read()
>>> w.transform = 'r'
>>> sm = Spatial_Markov(rpci, w, fixed=True, k=5)
>>> for p in sm.P:
...     print p
...     
[[ 0.96341463  0.0304878   0.00609756  0.          0.        ]
 [ 0.06040268  0.83221477  0.10738255  0.          0.        ]
 [ 0.          0.14        0.74        0.12        0.        ]
 [ 0.          0.03571429  0.32142857  0.57142857  0.07142857]
 [ 0.          0.          0.          0.16666667  0.83333333]]
[[ 0.79831933  0.16806723  0.03361345  0.          0.        ]
 [ 0.0754717   0.88207547  0.04245283  0.          0.        ]
 [ 0.00537634  0.06989247  0.8655914   0.05913978  0.        ]
 [ 0.          0.          0.06372549  0.90196078  0.03431373]
 [ 0.          0.          0.          0.19444444  0.80555556]]
[[ 0.84693878  0.15306122  0.          0.          0.        ]
 [ 0.08133971  0.78947368  0.1291866   0.          0.        ]
 [ 0.00518135  0.0984456   0.79274611  0.0984456   0.00518135]
 [ 0.          0.          0.09411765  0.87058824  0.03529412]
 [ 0.          0.          0.          0.10204082  0.89795918]]
[[ 0.8852459   0.09836066  0.          0.01639344  0.        ]
 [ 0.03875969  0.81395349  0.13953488  0.          0.00775194]
 [ 0.0049505   0.09405941  0.77722772  0.11881188  0.0049505 ]
 [ 0.          0.02339181  0.12865497  0.75438596  0.09356725]
 [ 0.          0.          0.          0.09661836  0.90338164]]
[[ 0.33333333  0.66666667  0.          0.          0.        ]
 [ 0.0483871   0.77419355  0.16129032  0.01612903  0.        ]
 [ 0.01149425  0.16091954  0.74712644  0.08045977  0.        ]
 [ 0.          0.01036269  0.06217617  0.89637306  0.03108808]
 [ 0.          0.          0.          0.02352941  0.97647059]]
\end{Verbatim} 
\end{footnotesize}
\end{frame}

\begin{frame}[fragile]
  \frametitle{Space-Time Rank Concordance}
  \begin{footnotesize}
  \begin{Verbatim}[numbers=left]
>>> import pysal
>>> f=pysal.open(pysal.examples.get_path("mexico.csv"))
>>> vnames=["pcgdp%d"%dec for dec in range(1940,2010,10)]
>>> y=np.transpose(np.array([f.by_col[v] for v in vnames]))
>>> regime=np.array(f.by_col['esquivel99'])
>>> w=pysal.weights.regime_weights(regime)
>>> np.random.seed(12345)
>>> res=[pysal.SpatialTau(y[:,i],y[:,i+1],w,99) for i in range(6)]
>>> for r in res:
...     ev = r.taus.mean()
...     "%8.3f %8.3f %8.3f"%(r.tau_spatial, ev, r.tau_spatial_psim)
...     
'   0.281    0.466    0.010'
'   0.348    0.499    0.010'
'   0.460    0.546    0.020'
'   0.505    0.532    0.210'
'   0.483    0.499    0.270'
'   0.572    0.579    0.280'
\end{Verbatim} 
\end{footnotesize}
\end{frame}

\begin{frame}[fragile]
  \frametitle{Side note: PySAL v. Scipy Kendall's Tau}
  \begin{footnotesize}
  \begin{Verbatim}[numbers=left]
# pvs.py
import pysal
import numpy as np
from scipy.stats import kendalltau
import time
ns = [25000, 50000, 100000, 200000]
for n in ns:
    x = np.arange(n); y = np.arange(n)
    y[40:50] = 20
    y = np.random.permutation(y)
    t1 = time.time(); res_p = pysal.Tau(x,y); t2 = time.time()
    res_s = kendalltau(x,y); t3 = time.time()
    print  'n=%d scipy/pysal: %8.3f'%(n, (t3-t2)/(t2-t1))

>>> run pvs.py
n=25000 scipy/pysal:    3.691
n=50000 scipy/pysal:    4.030
n=100000 scipy/pysal:    3.921
n=200000 scipy/pysal:    3.891
\end{Verbatim} 
\end{footnotesize}
\end{frame}
\begin{frame}\frametitle{Moran Scatter}

\begin{center}
     \begin{figure}
     \scalebox{0.75}{
     \includegraphics[width=1.0\linewidth]{moranscatter1} }
      \end{figure}      
\end{center}
  
\end{frame}


\begin{frame}\frametitle{Global and Local Autocorrelation}
  \begin{columns}
  \begin{column}[l]{3cm}
 
\begin{center}
     \begin{figure}
     \scalebox{1.00}{
     \includegraphics[width=1.0\linewidth]{moranscatter1} }
      \end{figure}      
\end{center}
 
  \end{column}
  \begin{column}[r]{7cm}
    Global Autocorrelation
    \begin{equation}
    I_t =  (n/S_0) \frac{\sum_i \sum_j z_{i,t} w_{i,j} z_{j,t}}{\sum_i
    z_{i,t}^2}
    \end{equation}

    Local Autocorrelation
    \begin{equation}
    I_{i,t} =  (z_{i,t}/m_2) \sum_j w_{i,j} z_{j,t}
  \end{equation}

  $m_2 = \sum_i z_{i,t}^2 / n$
  \end{column}
  \end{columns}
\end{frame}

\begin{frame}\frametitle{LISA Markov}
  \begin{columns}
  \begin{column}[l]{6cm}
 
\begin{center}
     \begin{figure}
     \scalebox{1.00}{
     \includegraphics[width=1.0\linewidth]{moranscatter1} }
      \end{figure}      
\end{center}
 
  \end{column}
  \begin{column}[r]{6cm}
 \begin{figure}
     \scalebox{1.00}{
     \includegraphics[width=1.0\linewidth]{moranscatter2} }
      \end{figure}      

  \end{column}
  \end{columns}
\end{frame}



\begin{frame}[fragile]
  \frametitle{LISA Markov}
  \begin{footnotesize}
  \begin{Verbatim}[numbers=left]
>>> import numpy as np
>>> f = pysal.open(pysal.examples.get_path("usjoin.csv"))
>>> pci = np.array([f.by_col[str(y)] for y in range(1929,2010)]).transpose()
>>> w = pysal.open(pysal.examples.get_path("states48.gal")).read()
>>> lm = LISA_Markov(pci,w)
>>> lm.classes
array([1, 2, 3, 4])
>>> lm.steady_state
matrix([[ 0.28561505],
        [ 0.14190226],
        [ 0.40493672],
        [ 0.16754598]])
>>> lm.p
matrix([[ 0.92985458,  0.03763901,  0.00342173,  0.02908469],
        [ 0.07481752,  0.85766423,  0.06569343,  0.00182482],
        [ 0.00333333,  0.02266667,  0.948     ,  0.026     ],
        [ 0.04815409,  0.00160514,  0.06420546,  0.88603531]])
\end{Verbatim} 
\end{footnotesize}
\end{frame}


\begin{frame}[fragile]
  \frametitle{LISA Markov - Spillover and Components}
  \begin{footnotesize}
  \begin{Verbatim}[numbers=left]
>>> lm_random = pysal.LISA_Markov(pci, w, permutations=99)
>>> r = lm_random.spillover()
>>> r['components'][:,12]
array([ 0.,  1.,  0.,  1.,  0.,  2.,  2.,  0.,  0.,  0.,  0.,  0.,  0.,
        0.,  0.,  0.,  0.,  2.,  2.,  0.,  0.,  0.,  0.,  0.,  0.,  1.,
        2.,  2.,  0.,  2.,  0.,  0.,  0.,  0.,  1.,  2.,  2.,  0.,  0.,
        0.,  0.,  0.,  2.,  0.,  0.,  0.,  0.,  0.])
>>> r['components'][:,13]
array([ 0.,  2.,  0.,  2.,  0.,  1.,  1.,  0.,  0.,  2.,  0.,  0.,  0.,
        0.,  0.,  0.,  0.,  1.,  1.,  0.,  0.,  0.,  0.,  0.,  0.,  2.,
        0.,  1.,  0.,  1.,  0.,  0.,  0.,  0.,  2.,  1.,  1.,  0.,  0.,
        0.,  0.,  2.,  1.,  0.,  2.,  0.,  0.,  0.])
>>> r['spill_over'][:,12]
array([ 0.,  0.,  0.,  0.,  0.,  0.,  0.,  0.,  0.,  1.,  0.,  0.,  0.,
        0.,  0.,  0.,  0.,  0.,  0.,  0.,  0.,  0.,  0.,  0.,  0.,  0.,
        0.,  0.,  0.,  0.,  0.,  0.,  0.,  0.,  0.,  0.,  0.,  0.,  0.,
        0.,  0.,  1.,  0.,  0.,  1.,  0.,  0.,  0.])

\end{Verbatim} 
\end{footnotesize}
\end{frame}




\begin{frame}\frametitle{CAST: Crime Analytics for Space-Time}
  \begin{block}{NIJ Project}
    \begin{itemize}
      \item Under development
      \item Stand-alone program based on PySAL's spatial dynamics module
      \item WxPython front-end
    \end{itemize}
  \end{block}
\end{frame}

\begin{frame}\frametitle{Linked Maps}
  
\begin{center}
     \begin{figure}
     \scalebox{0.75}{
     \includegraphics[width=1.0\linewidth]{linked_maps} }
      \end{figure}      
\end{center}
\end{frame}

\begin{frame}\frametitle{Linked Views}
  
\begin{center}
     \begin{figure}
     \scalebox{0.75}{
     \includegraphics[width=1.0\linewidth]{linked_views} }
      \end{figure}      
\end{center}
\end{frame}

\begin{frame}\frametitle{LISA Markov Plots}
  
\begin{center}
     \begin{figure}
     \scalebox{0.75}{
     \includegraphics[width=1.0\linewidth]{lisa_markov_plots} }
      \end{figure}      
\end{center}
\end{frame}


\subsection{Spatial  Econometrics}

\begin{frame}\frametitle{Spatial Econometrics}
  \begin{itemize}
    \item Still Rare in Commercial Econometrics Packages
      \begin{itemize}
        \item exception: Stata (ado files)
      \end{itemize}
    \item Many Specialized Scripts
    \begin{itemize}
        \item SAS, SPSS, Xlispstat, RATS, etc
    \end{itemize}
    \item Several Open Toolboxes
      \begin{itemize}
        \item spdep R, LeSage-Pace-Elhorst Matlab
      \end{itemize}
  \end{itemize}
  
\end{frame}

\begin{frame}\frametitle{PySAL: spreg}
  \begin{block}{Goals}
    \begin{itemize}
      \item Handle large problems
      \item Efficient spatial weights
      \item Modularity and reusability 
       \item Model complexity
    \end{itemize}
  \end{block}
\end{frame}
\begin{frame}\frametitle{Spatial Lag Model}
  \begin{block}{Specification}
\begin{equation}
\mathbf{y} = \rho \mathbf{W} \mathbf{y} + \mathbf{X} \beta + \mathbf{u},
\end{equation}
with $\mathbf{u} \sim N(0, \mathbf{\Sigma}_{\theta})$.
  \end{block}
  \begin{block}{Log Likelihood}
\begin{eqnarray}\label{eq:spatiallagloglik}
L &= &-(n/2)(\ln 2\pi) - (1/2) \ln | \mathbf{\Sigma}_{\theta} | + \ln | \mathbf{I} - \rho \mathbf{W} | \nonumber \\
     & &- (1/2)(y - \rho \mathbf{W y} - \mathbf{X} \beta)' \mathbf{\Sigma}_{\theta}^{-1}(\mathbf{y} - \rho \mathbf{W y} - \mathbf{X} \beta).
\end{eqnarray}
   \end{block}
   \begin{block}{Prediction}
\begin{equation}\label{eq:lagpredicted}
\hat{\mathbf{y}} = ( \mathbf{I} - \hat{\rho} \mathbf{W} )^{-1} \mathbf{X} \hat{\beta}.
\end{equation}
    \end{block}
\end{frame}
\begin{frame}\frametitle{spreg}
  \begin{block}{Functionality}
    \begin{itemize}
      \item OLS, Two Stage Least Squares
      \item Spatial Two Stage Least Squares
      \item GM Error (Kelejian and Prucha 98-99)
      \item GM Error Homoskedasticity (Drukker et al. 2010)
      \item GM Error Heteroskedasticity (Arraiz et al. 2010)
      \item Spatial HAC variance-covaraiance estimation
      \item Anselin-Kelejian tests for residual spatial autocorrelation from
        IV regression
      \item LM diagnostics spatial error and spatial lag
      \item Robust-LM diagnostics spatial error and spatial lag
      \item Probit
      \item Probit with spatial diagnostics
    \end{itemize}
  
  \end{block}
\end{frame}

%\begin{frame}\frametitle{GeoDaSpace}
%
%\begin{center}
%     \begin{figure}
%     \scalebox{0.65}{
%     \includegraphics[width=\linewidth]{space1.png} }
%     \caption{}
%      \end{figure}      
%\end{center}
%  
%\end{frame}
%
%\begin{frame}\frametitle{DBF reader}
%
%\begin{center}
%     \begin{figure}
%     \scalebox{0.65}{
%     \includegraphics[width=\linewidth]{space2.png} }
%     \caption{}
%      \end{figure}      
%\end{center}
%  
%\end{frame}
%
%
%\begin{frame}\frametitle{GAL reader}
%
%\begin{center}
%     \begin{figure}
%     \scalebox{0.65}{
%     \includegraphics[width=\linewidth]{space3.png} }
%     \caption{}
%      \end{figure}      
%\end{center}
%  
%\end{frame}
%
%  
%\begin{frame}\frametitle{Variable Selection}
%
%\begin{center}
%     \begin{figure}
%     \scalebox{0.65}{
%     \includegraphics[width=\linewidth]{space4.png} }
%      \end{figure}      
%\end{center}
%   
%\end{frame}

\begin{frame}\frametitle{OLS with Diagnostics}

\begin{center}
     \begin{figure}
     \scalebox{0.65}{
     \includegraphics[width=\linewidth]{space5.png} }
      \end{figure}      
\end{center}

\end{frame}

\begin{frame}\frametitle{Output}

\begin{center}
     \begin{figure}
     \scalebox{0.85}{
     \includegraphics[width=\linewidth]{spaceR.png} }
      \end{figure}      
\end{center}

\end{frame}

\begin{frame}\frametitle{Estimation: Spatial Lag Model}

\begin{center}
     \begin{figure}
     \scalebox{0.65}{
     \includegraphics[width=\linewidth]{space_lag.png} }
      \end{figure}      
\end{center}

\end{frame}

\begin{frame}\frametitle{Estimation: Spatial Lag Model}

\begin{center}
     \begin{figure}
     \scalebox{0.85}{
     \includegraphics[width=\linewidth]{spatial_lag_results} }
      \end{figure}      
\end{center}

\end{frame}



\begin{frame}[fragile]
  \frametitle{spreg}
  \begin{footnotesize}
  \begin{Verbatim}[numbers=left]
>>> from pysal.spreg import GM_Error_Het
>>> db = pysal.open('south.dbf', 'r')
>>> y = np.array(db.by_col('HR90'))
>>> y.shape = (len(y),1)
>>> X = []
>>> rhs = "RD90", "PS90", "DV90", "MA90", "UE90"
>>> tmp = [X.append(db.by_col(col)) for col in rhs]
>>> X = np.array(X).T
>>> w = pysal.queen_from_shapefile('south.shp')
>>> w.transform = 'r'
>>> reg = GM_Error_Het(y, X, w)
>>> reg.betas
array([[ 6.74524073],
       [ 4.41126522],
       [ 1.78024214],
       [ 0.48884888],
       [-0.01499858],
       [-0.39338759],
       [ 0.3132412 ]])
\end{Verbatim} 
\end{footnotesize}
\end{frame}


\begin{frame}
  
\begin{center}
     \begin{figure}
     \scalebox{0.95}{
     \includegraphics[width=1.0\linewidth]{spreg_diag} }
      \end{figure}      
\end{center}
\end{frame}

\begin{frame}
  
\begin{center}
     \begin{figure}
     \scalebox{0.95}{
     \includegraphics[width=1.0\linewidth]{spreg_est} }
      \end{figure}      
\end{center}
\end{frame}



\section{Future}
\subsection{Toolkits}

\begin{frame}\frametitle{ArcGIS Toolkit}
\begin{center}
     \begin{figure}
     \scalebox{0.95}{
     \includegraphics[width=1.0\linewidth]{10_arctool} }
      \end{figure}      
\end{center}
\end{frame}

\begin{frame}\frametitle{QGIS Toolkit}
\begin{center}
     \begin{figure}
     \scalebox{0.95}{
     \includegraphics[width=1.0\linewidth]{qgis_pysal} }
      \end{figure}      
\end{center}
\end{frame}

\begin{frame}\frametitle{For More Information}

  http://www.giscience.org/workshops.html
  
  http://pysal.org

  http://geodacenter.asu.edu

  email: srey@asu.edu
  
\end{frame}


\end{document}
